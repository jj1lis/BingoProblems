\documentclass[uplatex,a4paper,dvipdfmx]{jsarticle}
\usepackage{amsmath,amsfonts,amssymb,amsthm,bm,ascmac}
\usepackage{physics}
\usepackage{tikz}
\usepackage{gnuplot-lua-tikz,circuitikz}
\usetikzlibrary{intersections,calc,arrows.meta,cd,automata,positioning}
\usepackage{here,siunitx,multicol,multirow,systeme}
\usepackage[version=3]{mhchem}
\usepackage{chemfig,url,braket,enumerate,mathrsfs,otf,ulem,stmaryrd,titlesec}
\usepackage{listings,jlisting}
\lstset{
    language={C}, 
    backgroundcolor={\color[gray]{.85}},
    basicstyle={\footnotesize\ttfamily},
    identifierstyle={\footnotesize\ttfamily},
    commentstyle={\footnotesize\ttfamily \color[rgb]{0.5,0.5,0.5}},
    keywordstyle={\footnotesize\ttfamily \color[rgb]{0,0,0}},
    ndkeywordstyle={\footnotesize\ttfamily},
    stringstyle={\footnotesize\ttfamily}, 
    frame={tb}, 
    breaklines=true, 
    columns=[l]{fullflexible},
    numbers=left,
    %numbers=none, 
    xrightmargin=0zw,
    xleftmargin=3zw, 
    numberstyle={\scriptsize},
    stepnumber=1, 
    numbersep=1zw,
    escapechar={\^},
    morecomment=[l]{//}
} 
\renewcommand{\labelenumi}{(\arabic{enumi})}
\DeclareMathOperator{\Sinarc}{\mathrm{Sin}^{-1}}
\DeclareMathOperator{\Cosarc}{\mathrm{Cos}^{-1}}
\DeclareMathOperator{\Tanarc}{\mathrm{Tan}^{-1}}
\DeclareMathOperator{\Real}{\mathbb{R}}
\DeclareMathOperator{\Complex}{\mathbb{C}}
\DeclareMathOperator{\Rational}{\mathbb{Q}}
\DeclareMathOperator{\Natural}{\mathbb{N}}
\DeclareMathOperator{\Integer}{\mathbb{Z}}
\DeclareMathOperator{\Ker}{\mathrm{Ker}}
\DeclareMathOperator{\diffset}{\backslash}
\DeclareMathOperator{\define}{\overset{\text{def}}{\Longleftrightarrow}}

\theoremstyle{definition}
\newtheorem{dfn}{定義}
\newtheorem{axiom}{公理}
\newtheorem{thm}{定理}
\newtheorem{prop}{命題}
\newtheorem{lem}{補題}
\newtheorem{cor}{系}
\newtheorem{example}{例}
\newtheorem{question}{問題}
\newtheorem{caution}{注意}
\newtheorem{notation}{記法}


\title{$n$-ビンゴの臨界盤面数問題}
\author{}
\date{}

\newcommand{\bool}{\mathbb{B}}
\newcommand{\board}{\mathcal{B}}

\begin{document}
\maketitle
以下,$\bool := \qty{0, 1}$は通常のブール代数としての構造を持つものとする.また行ベクトルと列ベクトルは自然に同一視し,特に断りのない限り行ベクトルで表現する.
\begin{dfn}
    0でない自然数$n$に対して,\textbf{$n$-ビンゴ} ($n$-Bingo) とは以下に定義されるような対象の総称である.
    \begin{enumerate}
        \item \textbf{$n$-ビンゴ盤} ($n$-Bingo board) ないし単に\textbf{盤面} (board) とは,$\bool$上の$n \times n$行列のことである.
            $n$-ビンゴ盤の全体がなす集合を$\board_n := \bool^{n \times n}$とも表す.
        \item 盤面$B = [b_{i, j}]\in \board_n$と自然数$i$ ($1 \le i \le n$) に対して,
            \begin{itemize}
                \item $B$の第$i$\textbf{ヨコ列} (horizontal) $R_i(B)$とはいわゆる第$i$行,すなわちベクトル$(b_{i, 1}, \ldots, b_{i, n}) \in \bool^n$のことである.
                \item $B$の第$i$\textbf{タテ列} (vertical) $C_i(B)$とはいわゆる第$i$列,すなわちベクトル$(b_{1, i}, \ldots, b_{n, i}) \in \bool^n$のことである.
                \item $B$の\textbf{左ナナメ列} (left-diagnal) とは$D_l(B) := (b_{1, 1}, \ldots, b_{i, i}, \ldots, b_{n, n})$で表されるベクトル$D_l(B) \in \bool^n$のことである.
                \item $B$の\textbf{右ナナメ列} (right-diagnal) とは$D_r(B) := (b_{1, n}, \ldots, b_{i, n - i + 1}, \ldots, b_{n, 1})$で表されるベクトル$D_r(B) \in \bool^n$のことである.
                \item $B$の各ヨコ列,タテ列,左右ナナメ列を総じて$B$の\textbf{列} (line) といい,その集合を$L(B)$で表す.
            \end{itemize}
        \item 盤面$B \in \board_n$について,ある列$l \in L(B)$が存在して$l = (1, \ldots, 1)$が成り立つとき,またそのときに限り,$B$は\textbf{ビンゴ} (bingo) であるという.
    \end{enumerate}
\end{dfn}
\begin{notation}
    盤面$B \in \board_n$の$(r, c)$成分を$a \in \bool$に置き換えた盤面を$B_{[(r, c) \mapsto a]}$で表す.すなわち$B = [b_{i, j}]$とすれば,
    \begin{equation*}
        B_{[(r, c) \mapsto a]} = [a_{i, j}],\quad
        a_{i, j} = \left\{
            \begin{array}{cl}
                a &\text{if}\ (i, j) = (r, c) \\
                b_{i, j} & \text{otherwise}
            \end{array}
        \right.
    \end{equation*}
    と書くことができる.
\end{notation}
\begin{dfn}
    盤面$B = [b_{i, j}] \in \board_n$が\textbf{臨界} (critical) であるとは,$B$が次の条件をどちらも満たすことをいう.
    \begin{itemize}
        \item $B$はビンゴでない.
        \item $b_{i, j} = 0$であるような任意の$i, j$ ($1 \le i, j \le n$) に対して,$B_{[(i, j) \mapsto 1]}$がビンゴになる
    \end{itemize}
\end{dfn}

このとき,次のような問題を考えることができる.
\begin{screen}
\begin{question}[Kuwada, 2024] \label{question:kuwada}
    $n$-ビンゴの臨界盤面はいくつあるか.
\end{question}
\end{screen}
なお問題\ref{question:kuwada}(臨界盤面数問題)は未解決である.

\newpage

\begin{dfn}
    各0でない自然数$n$に対して,$\board_n$上の順序$\preceq$を次で定義する.
    \begin{quote}
        $A, B \in \board_n$に対して$0$個以上の組$(i_1, j_1), \ldots, (i_m, j_m)$が存在(ただし各$1 \le k \le m$について$1 \le i_k, j_k \le n$)して
    \end{quote}
\end{dfn}
\end{document}
